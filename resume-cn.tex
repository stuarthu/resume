\documentclass[11pt,oneside]{article}
\usepackage{geometry}
\usepackage{hyperref}
\usepackage[T1]{fontenc}

\usepackage{CJK}
\begin{CJK*}{UTF8}{gkai}

\pagestyle{empty}
\geometry{letterpaper,tmargin=1in,bmargin=1in,lmargin=1in,rmargin=1in,headheight=0in,headsep=0in,footskip=.3in}
\hypersetup{colorlinks=true,pdfstartview={FitH},urlcolor=red}

\setlength{\parindent}{0in} \setlength{\parskip}{0in}
\setlength{\itemsep}{0in} \setlength{\topsep}{0in}
\setlength{\tabcolsep}{0in}

% Name and contact information
\newcommand{\name}{胡世杰 (Stuart Hu)}
\newcommand{\phone}{Phone: 13917285414}
\newcommand{\email}{E-mail: \href{mailto:stuarthu@gmail.com}{stuarthu@gmail.com}}

%%%%%%%%%%%%%%%%%%%%%%%%%%%%%%%%%%%%%%%%%%%%%%%%%%%%%%%%%
% New commands and environments

% This defines how the name looks
\newcommand{\bigname}[1]{
	\begin{center}\selectfont\Huge\scshape#1\end{center}
}

% A ressection is a main section (<H1>Section</H1>)
\newenvironment{ressection}[1]{
	\vspace{4pt}
	{\selectfont\Large#1}
	\begin{itemize}
	\vspace{3pt}
}{
	\end{itemize}
}

% A resitem is a simple list element in a ressection (first level)
\newcommand{\resitem}[1]{
	\vspace{-4pt}
	\item \begin{flushleft} #1 \end{flushleft}
}

% A ressubitem is a simple list element in anything but a ressection (second level)
\newcommand{\ressubitem}[1]{
	\vspace{-1pt}
	\item \begin{flushleft} #1 \end{flushleft}
}

% A resbigitem is a complex list element for stuff like jobs and education:
%  Arg 1: Name of company or university
%  Arg 2: Location
%  Arg 3: Title and/or date range
\newcommand{\resbigitem}[3]{
	\vspace{-5pt}
	\item
	\textbf{#1}---#2 \\
	\textit{#3}
}

% This is a list that comes with a resbigitem
\newenvironment{ressubsec}[3]{
	\resbigitem{#1}{#2}{#3}
	\vspace{-2pt}
	\begin{itemize}
}{
    \end{itemize}
}

% This is a simple sublist
\newenvironment{reslist}[1]{
	\resitem{\textbf{#1}}
	\vspace{-5pt}
	\begin{itemize}
}{
	\end{itemize}
}

%%%%%%%%%%%%%%%%%%%%%%%%%%%%%%%%%%%%%%%%%%%%%%%%%%%%%%%%%
% Now for the actual document:

\begin{document}

% Name with horizontal rule
\bigname{\name}

\vspace{-8pt} \rule{\textwidth}{1pt}

\vspace{-1pt} {\small\itshape \phone \hfill \email}

\vspace{8 pt}


%%%%%%%%%%%%%%%%%%%%%%%%
\begin{ressection}{教育经历}

	\begin{ressubsec}{上海交通大学}{上海}{信息安全硕士}
		\ressubitem{2006.2 -- 2008.12}
	\end{ressubsec}

	\begin{ressubsec}{上海交通大学}{上海}{电子信息与信息工程学士}
		\ressubitem{1999.9 -- 2003.7}
	\end{ressubsec}

\end{ressection}

%%%%%%%%%%%%%%%%%%%%%%%%
\begin{ressection}{工作经历}

	\begin{ressubsec}{英礴}{上海,688广场}{资深架构师: 2019.5 -- 至今}
		\ressubitem{将SpatialOS从谷歌GKE移植到中国AWS上用kops搭建的K8S平台,包括容器服务ECR、对象存储S3、虚拟服务器EC2以及身份管理IAM全部切换至AWS平台并通过Terraform管理,DNS服务切换至Azure中国。}
		\ressubitem{在BuildKite上创建CICD pipeline,让中国区的生产和测试环境的rollout和发布自动化。}
		\ressubitem{在China Infra团队内进行知识分享,帮助团队了解OAUTH基本原理,go语言高阶应用等。}
		\ressubitem{架构并实现了中国区的账单服务平台。}
	\end{ressubsec}

	\begin{ressubsec}{七牛云}{上海,张江}{技术专家: 2018.4 -- 2019.5}
		\ressubitem{实现各种私有云NAS协议,包括NFS/FTP。}
		\ressubitem{开发七牛KODO对象存储的版本管理功能。}
		\ressubitem{开发七牛KODO对象存储的S3兼容网关,在原有的基础上完善了功能,包括完整的Service、Bucket、Object API,虚拟主机域名访问,AWS临时安全凭证等)并将性能提升至跟七牛KODO对象存储一致。}
		\ressubitem{在职期间为公司提交专利申请4份,申请号:201810813487.3、201811227621.8、201910124563.4、201910124576.1}
	\end{ressubsec}

	\begin{ressubsec}{EMC}{上海,创智天地}{软件主管工程师: 2014.2 -- 2018.4}
		\ressubitem{云存储服务器Infra团队开发Leader,主要工作内容包括:架构设计,评审成员代码,调查并解决生产环境中发生的问题。}
		\ressubitem{迅速掌握S3对象存储协议,Golang编程语言,Cloud Foundry开发部署平台,Cassandra数据库以及ElasticSearch,并能深入浅出地跟团队成员讲解技术细节。}
		\ressubitem{连续两年(2015,2016)在年终总评获得Consistently Exceeds评价。}
	\end{ressubsec}

	\begin{ressubsec}{野村信息}{上海,中环广场}{高级软件工程师: 2013.1 -- 2014.1}
		\ressubitem{对投行Position管理系统各个模块进行开发和维护。分析和解决各种技术问题。}
	\end{ressubsec}

	\begin{ressubsec}{SAP Labs China}{上海,张江}{高级软件工程师: 2007.9 -- 2013.1}
		\ressubitem{SAP Business One产品开发。本人所在Inventory Team负责库存模块的开发与维护。}
	\end{ressubsec}

	\begin{ressubsec}{智邦科技}{上海,漕河泾}{MPLS \& Multicast组 高级软件工程师: 2006.10 -- 2007.9}
		\ressubitem{开发、维护和测试混合交换机上的路由协议软件,包括MPLS,LDP以及DVMRP。操作系统是VxWorks,开发语言是C。}
	\end{ressubsec}

	\begin{ressubsec}{鹏越惊虹}{上海,张江}{软件工程师: 2003.7 -- 2006.10}
		\ressubitem{开发、维护和测试Linux上的网络安全软件。主要开发语言是C/C++,部分配置管理功能用Perl/Bash实现。}
	\end{ressubsec}

\end{ressection}

\begin{ressection}{技能}

	\begin{reslist}{研发能力:}
		\ressubitem{分布式对象存储系统专家,在该领域拥有多年的架构、开发、部署、解决线上问题的经验。}
		\ressubitem{熟悉软件开发流程。}
		\ressubitem{善于且乐于分析和解决问题。}
		\ressubitem{善于在技术层面指导团队成员,帮助解决技术难题。}
		\ressubitem{工作态度积极,能主动承担新的责任且乐于接受挑战。}
	\end{reslist}

	\begin{reslist}{英语:}
		\ressubitem{大学英语6级}
		\ressubitem{十年以上外企工作经历,熟练的口语和书面交流}
	\end{reslist}

	\begin{reslist}{操作系统:}
		\ressubitem{多年丰富的Linux研发经验,深入理解Linux进程间通信,系统调用等过程,对OS相关的内容如timer,中断处理,进程调度等有一定的了解。}
		\ressubitem{多年Windows下Visual Studio开发经验。}
	\end{reslist}

	\begin{reslist}{数据库:}
		\ressubitem{熟练的SQL/NOSQL编程技巧。}
		\ressubitem{熟悉Cassandra,ElasticSearch等NoSQL数据库。}
	\end{reslist}

	\begin{reslist}{计算机语言:}
		\ressubitem{精通C/C++/Golang。}
		\ressubitem{熟悉各种脚本语言Perl/Python/Ruby/Bash等。}
	\end{reslist}

\end{ressection}

\begin{ressection}{其他}

	\begin{reslist}{开源贡献:}
		\ressubitem{为rocksdb, cert-manager等多个开源项目贡献过代码。}
	\end{reslist}

	\begin{reslist}{写作出版:}
		\ressubitem{\href{https://item.jd.com/60796801160.html}{《分布式对象存储——原理、架构及Go语言实现》}}
		\ressubitem{\href{https://item.jd.com/12495484.html}{《分分布式缓存——原理、架构及Go语言实现》}}
	\end{reslist}

	\begin{reslist}{技术翻译:}
		\ressubitem{\href{https://www.epubit.com/book/detail/33175}{《JavaScript面向对象精要》}}
		\ressubitem{\href{https://www.epubit.com/book/detail/15126}{《Python和HDF5大数据应用》}}
		\ressubitem{\href{https://www.epubit.com/book/detail/14722}{《Python高性能编程》}}
	\end{reslist}

\end{ressection}

\end{CJK*}

\end{document}
